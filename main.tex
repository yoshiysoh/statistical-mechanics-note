\documentclass[a4paper,11pt]{article}

\usepackage{amsmath,mathtools}


\begin{document}

\title{A Note on Statistical Mechanics}
\author{Yoshiaki Horiike}
\date{\today}
\maketitle
\begin{abstract}
  This article briefly reviews the derivation of important formulae of 
  microcanonical and canonical ensemble.
\end{abstract}

\section{Microcanonical Ensemble}
The microcanonical system is an isolated system that does not exchange 
energy (or particle) outside of the system.
Microcanonical ensemble assumes that every state with same
energy $E$ happens equally likely.
Therefore, $P^\mathrm{M}_i$, the probability\footnote
{
  This M stands for ``Microcanonical''.
}
which a (quantum) microstate $i$ happens, is given 
as a fraction of $W(E)$, the total number of microstates with energy $E$.
\begin{equation*}
  P^\mathrm{M}_i = \frac{1}{W(E)}
\end{equation*}
Ludwig Boltzmann defined the entropy $S$ using this $W(E)$.
\begin{equation*}
  S(E)
  \coloneqq
  k_\mathrm{B} \ln W(E)
\end{equation*}
Here, $k_\mathrm{B}$ is the Boltzmann constant.
The temperature of this system is defined as the partial differentiation 
of entropy $S$ by energy $E$.
\begin{equation*}
  \frac{1}{T}
  \coloneqq
  \frac{\partial S(E)}{\partial E}
\end{equation*}

\section{Canonical Ensemble}
Consider a subpart of the microcanonical system. 
This small subsystem can exchange energy with the rest of whole 
system. 
From below of this section, ``system'' means this small subsystem, 
``bath''\footnote{
  One can also say ``reservoir''.
} means the total system except for the small subsystem, 
``total system'' means a whole system consisting of the system and 
the bath.

Suppose that the system is in a state $i$ with energy $E_i$.
$E^\mathrm{bath}$, the energy of the bath is equall to
total energy $E^\mathrm{tot}$ without the system's energy.
\begin{equation*}
  E^\mathrm{bath}
  =
  E^\mathrm{tot}
  -
  E_i
\end{equation*}
By defining 
\begin{itemize}
  \item $W(E_i)=1$ as the number of states\footnote{We ignored the situation of degeneracy but the essentials do not change.} of the system with energy $E_i$, 
  \item $W(E^\mathrm{bath})$ as the number of states of the bath with energy $E^\mathrm{bath}$,
  \item $W(E^\mathrm{tot})$ as the number of states of the total system with energy $E^\mathrm{tot}$,
\end{itemize}
we can calculate $P^\mathrm{C}_i$, the probability\footnote
{
  This C indicates ``Canonical''.
} 
that system's state is in $i$ with energy $E_i$ as below.
\begin{align*}
  P^\mathrm{C}_i
  =&
  \frac{ W(E_i) W(E^\mathrm{bath}) }{ W(E^\mathrm{tot}) }
  \\=&
  \frac{ W(E^\mathrm{tot} - E_i) }{ W(E^\mathrm{tot}) }
  \\=&
  \exp 
  \left[
    \frac{S(E^\mathrm{tot} - E_i) - S(E^\mathrm{tot})}{k_\mathrm{B}} 
  \right]
\end{align*}
Thus, $P^\mathrm{C}_i$ depends on the bath's number of the states.
We perform several approximations.
\begin{align*}
  S(E^\mathrm{tot} - E_i)
  =&
  S(E^\mathrm{tot} - \left<E\right> + \left<E\right> - E_i)
  \\\approx&
  S(E^\mathrm{tot} - \left<E\right>)
  +
  \left.
  \frac{\partial S}{\partial E}
  \right|_{E = E^\mathrm{tot} - \left<E\right>}
  (\left<E\right> - E_i)
  \\\approx&
  S(E^\mathrm{tot} - \left<E\right>)
  +
  \frac{1}{T^\mathrm{bath}}
  (\left<E\right> - E_i)
\end{align*}
We assumed 
\begin{itemize}
  \item $E^\mathrm{tot} - \left<E\right> \gg \left<E\right> - E_i$ 
  \item
$
  \left.
  \frac{\partial S}{\partial E}
  \right|_{E = E^\mathrm{tot} - \left<E\right>}
  \approx
  \left.
  \frac{\partial S}{\partial E}
  \right|_{E = E^\mathrm{tot} - E_i}
  =
  \frac{1}{T^\mathrm{bath}}
$
\end{itemize}
in the process of approximations.

From the definition of the temperature of a microcanonical ensemble, 
$
  \frac{1}{T}
  \coloneqq
  \frac{\partial S(E)}{\partial E}
$,
we can split $S(E^\mathrm{tot})$ into two terms\footnote
{
  We can confirm this by a simple example.
  Consider the two variables $x$ and $y(x)$. 
  If we know the relation $\frac{\partial y(x)}{\partial x}=a$ 
  ($a$ is constant),
  the two varialbes relation is $y(x) = ax+b$ ($b$ is constant).
  We can write $y(x-\Delta x)$ as below.
  \begin{equation*}
    y(x-\Delta x)
    = 
    a(x-\Delta x)
    = 
    ax-a\Delta x
    =
    y(x) - y(\Delta x)
  \end{equation*}
  Thus, $y(x) = y(x - \Delta x) - y(\Delta x)$.
}.
\begin{equation*}
  S(E^\mathrm{tot})
  =
  S(E^\mathrm{tot}-\left<E\right>)
  +
  S(\left<E\right>)
\end{equation*}

Let's use those approximations in our calculation of $P^\mathrm{C}_i$.
\begin{align*}
  P^\mathrm{C}_i
  =&
  \exp 
  \left[
    \frac{S(E^\mathrm{tot} - E_i) - S(E^\mathrm{tot})}{k_\mathrm{B}} 
  \right]
  \\\approx&
  \exp 
  \left[
    \frac
    {
      S(E^\mathrm{tot} - \left<E\right>)
      +
      \frac{1}{T^\mathrm{bath}}
      (\left<E\right> - E_i)
      -
      S(E^\mathrm{tot}-\left<E\right>)
      -
      S(\left<E\right>)
    }
    {k_\mathrm{B}} 
  \right]
  \\=&
  \exp 
  \left[
    \frac
    {
      \left<E\right> 
      - 
      E_i
      -
      T^\mathrm{bath}S(\left<E\right>)
    }
    {k_\mathrm{B} T^\mathrm{bath}} 
  \right]
  \\=&
  \exp 
  \left[
    \frac
    {
      \left<E\right> 
      -
      T^\mathrm{bath}S(\left<E\right>)
    }
    {k_\mathrm{B} T^\mathrm{bath}} 
  \right]
  \exp 
  \left[
    -
    \frac
    {E_i}
    {k_\mathrm{B} T^\mathrm{bath}} 
  \right]
  \\=&
  \exp 
  \left[
    \frac
    {F}
    {k_\mathrm{B} T^\mathrm{bath}} 
  \right]
  \exp 
  \left[
    -
    \frac
    {E_i}
    {k_\mathrm{B} T^\mathrm{bath}} 
  \right]
  \\=&
  \frac{1}{Z}
  \exp 
  \left[
    -
    \frac
    {E_i}
    {k_\mathrm{B} T^\mathrm{bath}} 
  \right]
  \\\propto&
  \exp 
  \left[
    -
    \frac
    {E_i}
    {k_\mathrm{B} T^\mathrm{bath}} 
  \right]
\end{align*}
We defined several new quantities. 
\begin{itemize}
  \item $\left<E\right> 
  \coloneqq 
  \sum_i P^\mathrm{C}_i E_i$ is average energy of the system.
  \item $F 
  \coloneqq 
  \left<E\right> 
  - T^\mathrm{bath}S(\left<E\right>)$ is free energy.
  \item $Z \coloneqq \exp
  \left[
    -
    \frac
    {F}
    {k_\mathrm{B} T^\mathrm{bath}} 
  \right]$ is the partition function.
\end{itemize}
From the definition of a partition function above, 
we can obtain free energy as a function of partition function.
\begin{equation*}
  F \coloneqq - k_\mathrm{B}T \ln Z
\end{equation*}
From the normalization condition of probability $P^\mathrm{C}_i$, 
we can derive another form of a partition function.
\begin{align*}
  1 
  =& 
  \sum_i 
  P^\mathrm{C}_i
  \\=&
  \sum_i 
  \frac{1}{Z}
  \exp 
  \left[
    -
    \frac
    {E_i}
    {k_\mathrm{B} T^\mathrm{bath}} 
  \right]
  \\=&
  \frac{1}{Z}
  \sum_i 
  \exp 
  \left[
    -
    \frac
    {E_i}
    {k_\mathrm{B} T^\mathrm{bath}} 
  \right]
\end{align*}
Hence, we define a partition function as a normalization factor of 
probability.
\begin{equation*}
  Z 
  \coloneqq 
  \sum_i 
  \exp 
  \left[
    -
    \frac
    {E_i}
    {k_\mathrm{B} T^\mathrm{bath}} 
  \right]
\end{equation*}


\end{document}